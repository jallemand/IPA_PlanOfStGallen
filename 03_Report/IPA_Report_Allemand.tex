% !TeX document-id = {c744144e-6092-43bd-94df-1ae69bd94f6b}
% !TeX spellcheck = en_GB
% !TeX encoding = UTF-8
% !TeX TXS-program:compile = txs:///latexmk/[-pdf -silent -shell-escape -latexoption="-synctex=1" -output-directory="build" -r "docstyle/nomenclature_latexmkrc"]
% !TeX TXS-program:quick = txs:///compile | txs:///view

\documentclass[ipa,teamwork]{docstyle/IDSCreport}

\title{Title}
\subtitle{Subtitle}
\authorA{Jonathan Allemand}
\authorB{Sabine Rüdisühli}
\ethidA{17-937-632}
\ethidB{17-???-???}
\semesterA{HS 2019}
\semesterB{HS 2019}
\emailA{jonal@student.ethz.ch}
\emailB{sabiner@student.ethz.ch}
\supervision{Prof. Konrad Schindler and Cornel Dora}
\identification{IGP-XX-YY-ZZ}
\date{31st of May, 2019}
\keywords{photometric stereo, Plan of St Gall, feature detection, image stitching}

%\bibliography{bibliography}
%\maketitle

\begin{abstract}
%The start of the painting of the Plan of Saint Gall was in 16xx and afterwards, new parts were added. Due to the lifetime and the painting the plan gets some “injuries”. To detect traces of the past, the Plan was recorded with the best measurement system nowadays, the Minidome, which allows to measure with mm-submilitre resolution and in 2.5D. 
%
%To subtract some information from the Plan, firstly, the patches recording have to be stitched together. This steps have to be done because the portable Minidome can only record patches of a size of x X x cm and the Plan has a totally size of  x m. For the extracting of research features, ideas have to build up which should work on an old, crumbled plan. These detected features will be afterwards analysed from plan experts. 
%To prepare information for the experts, the plan was stitched together with Photoshop because all other program reached their limits with the given 1.5 TB dataset. The key point in this step was to get the transform parameters for each patch. After a lot of tries, Finally, a self-written C++ script solved the program. The second challenge, extracting research features like needle holes and scratches, can be only solved with manual detecting because the crumbled old plan destroyed all the genius, theoretical ideas for detecting. For examples, the made assumption that needle holes should be round and have some height differences are logical, but the circle matching program gave a lots of more possible circles which lays in wrinkled regions.\\
\end{abstract}

\begin{acknowledgement}
%\noindent We would like to sincerely thank Professor Schindler, because he made possible an extraordinary project that brought together our acquired technical knowledge and a cultural asset. In addition, for the simple, but still very good care. Furthermore, we sincerely thank the whole Stiftsbibliothek of St. Gall, who received us a very warm welcome and a great confidence to work with the unique plan. Very big help was the Abbey librarian Cornel x, who made all the impossible things possible, and Silvio y, who was always available for our questions and made the whole measuring process possible. Finally, we thank the Belgium team from Leuven, Vincent e and c.c, who made the recording with their brought minidome. 
\end{acknowledgement}

\begin{document}
% !TeX spellcheck = en_UK
% !TeX encoding = UTF-8
% !TeX root = ../report.tex

\chapter{Introduction}
\label{chp:IntroductionChapter}


\section{Plan of St Gall}
The Plan of St Gall is one of the only remaining major architectural drawings from the period between the fall of the Western Roman empire until the 13th century (Reference wikipedia...). At nearly 1200 years of age and consisting of 5 pieces of parchment stitched together, significant care has to be taken to ensure that it is preserved into the future. For this reason, there has been expressed interested in capturing 3D information about the surface of the plan at a fine scale that may not have been readily visible by the human eye. By capturing this information digitally, it enables researchers, historians or other interested parties unrestricted access to the plan from wherever they are in the world. Furthermore, in the mission to preserve the plan, this will reduce the need for physical inspection of the plan and decrease the amount it is exposed to environmental conditions that will hasten the deterioration process of the plan.

\section{Project Overview/goals}
Previously, a high resolution photogrammetric stereo 3D model has been generated of the Plan of St Gall, but similarly to the Nyquist-Shannon sampling theorem for signal processing, the resolution of the 3D model has to be higher than that of the features one seeks to identify within it. The 3D model acquired, although of a high resolution, was not quite high enough to discern fine features in the parchment, such as needle holes, without having the physical parchment for comparison or a prior knowledge of the existing feature. 

Because of this, it was proposed and accepted to perform an extremely high resolution photometric stereo capture of the full plan. Following the capture of the plan, the data acquired was to be combined into a full high resolution model or image mosaic.

In addition to the stitching of the plan, further exploratory analysis would be performed on the output. This includes feature detection of pinholes or scratch marks that have been expressed as of interest. Geometric and radiometric differences will also be analysed for various effects.

\section{Photometric Stereo}
Although photometric stereo does not natively provide a 3D model in the typical sense, but what it does capture is a set of 3D vectors representing the surface of the object projected onto an image plane. Thus the output is in the form of an image where the typical RGB colour channels representing a 3D normal vector for the area of the object that each pixel projected onto. This leads to the common reference of this being referred to as 2.5D data as the image only has 2 dimensions but the normal maps provide a discretised approximation of the normal vectors for the objects surface within the region each pixel covered. 

In addition to the 3D properties of the plan that can be obtained through the normal maps from the photometric stereo, ambient and albedo images can also be generated which can also be used to provide additional information about the surface of the objects reflectance properties which is not captured in a photogrammetric stereo model. 

The resulting project of the collaboration between the Stiftsbibliothek and the group behind the Minidome from Leuven University (check names) can be described in three parts. The first part consists of the preparation and data capture of the Plan of St Gall largely performed by the staff from Leuven. The last two sections represent the individual works of the two authors regarding the specific works each author performed on the Plan of St Gall.


% !TeX spellcheck = en_GB
% !TeX encoding = UTF-8
% !TeX root = ../report.tex

\chapter{Photometric Stereo Survey}
\label{chp:PhotometricStereo}

\section{Introduction}
This section will describe the main physical processes and considerations regarding the data acquisition of the Plan of St Gall using the Minidome. This chapter will not provide an in depth explanation of photometric stereo, nor some of the finer details associated to the Minidome that as volunteered and controlled by the group from Leuven as the major part of this project was focussed on utilising the post-processed outputs from the Minidome provided by them.

\section{Location}
As the Stiftsbezirk St.Gallen are the custodians of the Plan and responsible for the safety and security of the Plan, the photometric stereo survey was performed within a secure room provided on the premises. This allowed allowed for not only the secure storage of the Plan and Minidome equipment overnight, but also the oversight of their staff for the handling of the Plan throughout the data acquisition.

\section{Equipment}

- Minidome
	- 270 LEDs Whitelight
	- 180 Multispectral
	
- Camera specs?

- 1 plan

- Tripods and rail setup for holding the minidome over the plan

- Trolley

- Tables which the trolley would be rolled along

- Tape to mark positions of trolley to ensure full coverage

- Measuring "tape"/"stick" (I'm not sure what the measuring things here are called... I'm used to the metal rolling ones

- Laptop and NES storage system for capture of all the data

\section{Considerations}

- Room selection

- White light / Multispectral

- Setup and movement of the plan

- overlap

- Room lighting

- Room temperature

- Moving of floor boards
	- During acquisition
	
	
\section{Methodology}

	\paragraph{Acquisition}
	
	- Geometric calibration
	
	- Exposure calibration
	
	- Acquisition
	
	- White front
	
	- White back
	
	- Multispectral front
	
	- Limitation of NUV / NIR exposure to plan
		- Cover	
	

	\paragraph{Processing}
	
	- Camera calibration
	
	- Exposure calibration
	
	- Determine albedo / ambient / normal products
	
	- Cluster processing for exposure correction

\section{Results}

\section{Analysis}

% !TeX spellcheck = en_GB
% !TeX encoding = UTF-8
% !TeX root = ../report.tex

\chapter{Feature Extraction}
\label{chp:FeatureExtractionChapter}

Something Something Feature Extraction
% !TeX spellcheck = en_US
% !TeX encoding = UTF-8
% !TeX root = ../report.tex

\chapter{Plan Stitching}
\label{chp:MosaicStitchingChapter}

\section{Introduction}
The digital recreation of the plan consisted of two main parts. Firstly, there was the combination

	\paragraph{General Process}

\section{PImage Stitching}

	\subsection{Hugin}
	- Powerful open source software
	
	- Control nearly all aspects of the stitching workflow
		- Control point detection
		- Control point matching
		- Transformation parameter optimisation
		- Exposure compensation
		- Camera calibration
		- Varying projection options

	\subsection{Photoshop}
	- Easy to use
	- Minimal choice of parameters. Only a couple of projection choices and slight options for vignetting / geometric/exposure settings
	- License required
	- Processing time 30 min - 1 Hr

	\subsection{OpenCV}
	- Open source (free)
	- Full control of parameters
	- Ability to obtain optimised parameters - e.g. orientation and translations of images
	- Slow processing time... unable to obtain a full resolution output with existing class or settings?
	(Need to try again).

	\subsection{ICE}
	- Simple to use software
	- Free 
	- Some parameter control
		- More projection options than photoshop
	- Easy layout tools to achieve a better a priori of the relative image locations
	- Fast
	
	\subsection{Results}
	- Photoshop provided some of the most visually appealing results.
		- Line work and visual features were nearly all continuous with minimal discontinuities compared to others
	- Hugin could be consistent with the remapping and projection of results
		- Line control points provided additional constraints to the outside of the plan to ensure they were "straight" and not so affected by projective distortions
	- Ice was extremely fast and easy... although issues with the layout option for the white light as the software was rigid that all row and columns must contain the same number of images
	- Results were visually appealing... Equal with photoshop?

	\subsection{Discussion}

\section{3D Model Reconstruction}

	\subsection{Normal Map to Point Cloud}
	Normal maps, or gradient fields, are an approximation of an objects surface normal vector for the area covered by the pixel in the image. The normal vector doesn't provide any direct information about the height of the objects surface but it does provide the directional change in height - proportional to the width of a pixel.
	
	By knowing these height difference across the discretised pixels, it is possible to determine a height at each pixel point via integration. A simplistic approach of picking one pixel as the reference point and then performing line integrations across the normal map is an easy to implement solution but does not have a unique solution. Further methods aim to improve the overall solution through the likes of error minimisation such as global least squares or multigrid methods.
	
	For this project, two methods of integration were tested. A global least squares method was implemented using that proposed by (reference) in code develepoed by (reference). The second method was through a multigrid method implemented by ESAT at Leuven.
	
	\subsubsection{Global Least Squares}
	The global least squares method aims to be an improvement on a typical line integration by obtaining a unique solution. The implementation used by (reference) in their grad2surf() functions does .....
	
	As you can see in the Figure X, the result performs OK for a single patch.
				
	\subsection{Point Cloud Registration}
	
	\subsection{Point Cloud Cleaning}
	
	\subsection{Results}
	
	\subsection{Discussion}

% !TeX spellcheck = en_GB
% !TeX encoding = UTF-8
% !TeX root = ../report.tex

\chapter{Analysis}
\label{chp:AnalysisChapter}


% !TeX spellcheck = en_GB
% !TeX encoding = UTF-8
% !TeX root = ../report.tex

\chapter{Conclusion}
\label{chp:ConclusionChapter}

And the main conclusions to be put here.

\appendix
% !TeX spellcheck = en_US
% !TeX encoding = UTF-8
% !TeX root = ../report.tex

\chapter{Example Appendix Chapter}
\label{chp:ExampleAppendixChapter}

The following code is the definition of the bibliography entry of the document class IDSCreport~\cite{IDSCreportClass}.

\lstinputlisting[style=plaincode,xleftmargin=1em]{bibliography.bib}

\end{document}
